\documentclass[11pt]{beamer}
\usepackage[utf8]{inputenc}
\usepackage[T1]{fontenc}
\usepackage{lmodern}
\usepackage{listings}
\usetheme{Berlin}
\usepackage{hyperref}
\begin{document}
	\author{GROUPE 2}
	\title{M2 SLED 2023/2024}
	\subtitle{}
	%\logo{}
	\institute{}
	\date{\tiny Feb 29,  2024}
	\subject{}
	%\setbeamercovered{transparent}
	%\setbeamertemplate{navigation symbols}{}
	\begin{frame}[plain]
		\title{ {\small Problème du sac a dos avec Méthode de recherche locale itérée (Iterated local search)}}
		\author{{\tiny MASTER II SYTEMES ET LOGICIELS EN ENVIRONNEMENT DISTRIBUE(SLED)}\\ {\tiny Groupe 2}\\ {\tiny ENCADREUR: Pr. NDAM } }

		\begin{center}	
			{\footnotesize	UNIVERSITE DE NGAOUNDERE }	\\
		 {\tiny FACULTE DES SCIENCES }\\
		 {\tiny DEPARTEMENT DE MATHEMATIQUES ET INFORMATIQUE}\\
	 		\includegraphics[scale=0.1]{image}\\
	 \end{center}
		\maketitle
		 
	\end{frame}
	
	\begin{frame}
		\frametitle{Plan du travail}
		\tableofcontents
		\section{Introduction}
		\section{Methode de Rechercher Locale Itérée}
		\section{Implementation de la Recherche Locale Itérée}
		\section{Etude Comparative}
		\section{Conclusion}
	\end{frame}

% Section 1: Introduction
\section{Introduction}
\begin{frame}{Introduction}
    \begin{itemize}
        \item Présentation du problème du sac à dos
        \item Importance de trouver une solution optimale
        \item Présentation des méthodes de recherche locale
    \end{itemize}
\end{frame}

% Section 2: Méthode de Recherche Locale Itérée
\section{Méthode de Recherche Locale Itérée}
\begin{frame}{Méthode de Recherche Locale Itérée}
    \begin{itemize}
        \item Concept et principes de base
        \item Comparaison avec la méthode de Hill Climbing
        \item Avantages et inconvénients
    \end{itemize}
\end{frame}

% Section 3: Implémentation de la Recherche Locale Itérée
\section{Implémentation de la Recherche Locale Itérée}
\begin{frame}{Implémentation de la Recherche Locale Itérée}
    \begin{itemize}
        \item Description de l'algorithme
        \item Choix des paramètres
        \item Pseudo-code de l'algorithme
    \end{itemize}
\end{frame}

% Section 4: Étude Comparative
\section{Étude Comparative}
\begin{frame}{Étude Comparative}
    \begin{itemize}
        \item Étude de cas avec des exemples numériques
        \item Analyse des résultats obtenus
        \item Discussion sur l'efficacité et la robustesse de chaque méthode
    \end{itemize}
\end{frame}

% Section 5: Critères d'Acceptation et de Rejet des Nouveaux Optima
\section{Critères d'Acceptation et de Rejet des Nouveaux Optima}
\begin{frame}{Critères d'Acceptation et de Rejet des Nouveaux Optima}
    \begin{itemize}
        \item Importance de la perturbation dans la recherche locale itérée
        \item Réglage de la pression de perturbation
        \item Impact sur l'intensification et la diversification de la recherche
    \end{itemize}
\end{frame}

% Section 6: Conclusion
\section{Conclusion}
\begin{frame}{Conclusion}
    \begin{itemize}
        \item Résumé des points abordés
        \item Perspectives d'amélioration et de recherche future
    \end{itemize}
\end{frame}

% Slide pour les codes à présenter
\section{Codes}
\begin{frame}[fragile]{Codes}
    \begin{lstlisting}[language=Python]
        # Votre code Python ici
    \end{lstlisting}
\end{frame}
\begin{frame}
\begin{minipage}{0.48\textwidth}
\centering
\includegraphics[width=\textwidth]{merci2.png}
\end{minipage}\hfill
\begin{minipage}{0.48\textwidth}
\centering
\includegraphics[width=\textwidth]{question.png}
\end{minipage}
\end{frame}
\end{document}